\section{Introduction}

% ------------------- Introduce Bayes --------------------------

Accurate revenue forecasting plays a pivotal role in the success of
businesses, aiding in well-informed decision-making in many areas, like
resource allocation and planning. This study delves into applying Bayesian
models for predicting daily revenue at Copenhagen-based poke bowl chain;
OLIOLI. The comparison of various Bayesian models and their predictive
performance provides insights into the advantages and limitations of these
methods in a real-world context and supports ongoing research in Bayesian data
analysis and its application in industry.

Grounded in the principles of Bayes' theorem, Bayesian modelling provides an
alternative to the classic frequentist approach to statistical data analysis
and modelling. Bayesian methods incorporate prior knowledge and facilitate an
intuitive understanding of uncertainty. Both approaches can quantify
uncertainty, but the Bayesian framework inherently incorporates it, while the
frequentist approach requires additional measures such as p-values or
confidence intervals \cite{statrethinking}. This intuitive understanding and direct
representation of uncertainty in Bayesian methods make them particularly useful
for complex real-world challenges.

% ------------------- Define the problem --------------------------
This study applies Bayesian methods to a case study, illustrating their
potential for predictive modelling. Additionally, an examination is conducted
on how decisions made at various stages - from data processing and modelling to
model comparison - can influence the outcomes.

Selecting relevant, measurable variables for revenue modelling is
challenging due to the complexity of real-world phenomena. Store revenue,
for instance, is a composite of countless individual human decisions.
Therefore, reliance is placed on the domain knowledge of industry experts
who understand the temporal factors influencing revenue and can interpret
the daily, weekly, and seasonal patterns that are critical to business
performance.

One such factor these experts have identified is the weather. This
study is mainly motivated by the potential of weather phenomena as a
source of variability in revenue. To investigate this, the impact of various
weather conditions on revenue is studied, providing an insightful exploration
of the application of Bayesian methods in a real-world context.

% ------------------ Research Questions ----------------------------

In light of these approaches, the research questions that will be addressed in
this paper are:

\begin{enumerate}
    \item What is the best-performing Bayesian model for predicting daily
      revenue for OLIOLI using weather data?
    \item What is the optimal data volume for model performance in accurate revenue prediction?
    \item What are the assumptions and compromises of different models, and how
      do they impact performance and the interpretability of the results?
\end{enumerate}

Multiple Bayesian models and evaluation metrics will be tested,
compared, and evaluated to address these questions using real-world
sales data from OLIOLI. Current research in the field exists for
motivating the deployment of Bayesian models, with methodologies
existing for their evaluation and comparison \cite{gelman}
\cite{statrethinking} \cite{puppies}. Despite this, the exact process
for selecting the optimal model and data for a specific problem
remains unclear and context-dependent.

This paper consists of five further sections. First, background theory is
introduced to provide a foundation of knowledge from which the Bayesian
analysis of the case study can be understood. Second, the case study is
introduced, and the data is investigated. Third, the methodology motivates and
describes the models while reflecting on the initial predictive performance of
the various approaches applied to the case study. Fourth, the results are fully
presented, discussed, and the Bayesian models are compared. Finally, the paper
concludes with a summary of the findings and a discussion of the limitations
and future work of the study.

% ------------------ Contributions -------------------------------

% Thorough testing, evaluation, and comparison of different assumptions in Bayesian data analysis
% Focus on revenue prediction for a restaurant chain
% Potential for accurate forecasting, better decision-making, and efficient resource allocation
% The contributions of this paper are manifold, providing a comprehensive
% testing, evaluation, and comparison of the different assumptions within the
% realm of Bayesian data analysis; through the vessel of the mentioned case
% study. By investigating the research questions and evaluating the performance
% of various Bayesian models, this study aims to not only demonstrate the
% potential for accurate forecasting but also emphasise the potential benefit of
% informed decision-making and efficient resource allocation in a business
% context.
% Hierarchical extension possibilities
The contributions of this paper include the following:
\begin{itemize}
  \item Comprehensive testing, evaluation, and comparison of different
    Bayesian models in the context of revenue forecasting using real-world
    data. 
  \item An in-depth exploration of the assumptions inherent to Bayesian data
    analysis and their impact on model performance. 
  \item Demonstrating the potential for accurate forecasting through
    Bayesian methods, which can have significant implications for business
    decision-making and resource allocation.
\end{itemize}
