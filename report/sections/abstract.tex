\begin{abstract}
This study investigates using Bayesian models for daily revenue forecasting on
the case study of OLIOLI, a Copenhagen-based poke bowl chain. A comprehensive
assessment of various Bayesian models reveals their predictive strengths and
limitations within a real-world context. The research examines the influence of
weather indicators, such as maximum temperature and total precipitation, on
revenue. In the case study, Bayesian time-series models outperform their linear
regression counterparts, and weather factors are not found to add significant
value in models, treating revenue as a time-series. This challenges expert
assumptions about the causal relationship between weather and revenue. Offering
a thorough evaluation of Bayesian models for revenue prediction, this study
invites further inquiry into the complex relationships between environmental
factors and daily revenue for this case. It provides practical insights for revenue
forecasting for this case study and a critical perspective on Bayesian data
analysis assumptions and implications.
\end{abstract}
