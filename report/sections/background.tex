\section{Background}

% ------------------- Introduce Bayes --------------------------

\subsection{Bayesian Statistics}

This paper does not aim to serve as a guide to Bayesian statistics, nor an
explicit introduction to the theory. However, some background knowledge helps
with understanding the theory and assumptions motivated throughout the paper.


Introduction to Bayesian statistics: Begin by briefly introducing Bayesian statistics and emphasizing the key differences between the Bayesian and frequentist approaches. Mention that Bayesian statistics is a method for updating beliefs based on observed data, whereas frequentist statistics focuses on the likelihood of observing data given fixed model parameters.

Bayes' theorem: Introduce Bayes' theorem and explain how it provides a mathematical framework for updating prior beliefs with observed data. You could present the formula for Bayes' theorem and discuss its components, such as the prior, likelihood, and posterior.

Key concepts in Bayesian modeling: Discuss the basic concepts of Bayesian modeling, such as prior distributions, which represent initial beliefs about model parameters; likelihood functions, which describe the probability of observing data given the model parameters; and posterior distributions, which combine the prior and likelihood to obtain updated beliefs about the parameters.

Bayesian linear regression: Explain the concept of Bayesian linear regression and how it differs from traditional linear regression. You could mention that in Bayesian linear regression, model parameters are treated as random variables with their own prior distributions, and that uncertainty in the parameters is taken into account when making predictions.

Generalized linear models (GLMs): Briefly introduce generalized linear models and mention that they are a more general class of models that includes Bayesian linear regression as a specific example. Explain that GLMs can be used to model a wider range of relationships between predictors and response variables, and that they can accommodate different types of response distributions, such as Poisson or logistic distributions.



-- How do people already approach the problem 

-- Why is Bayesian modelling a good approach for this problem 

-- How can we evaluate Bayesian Models in this context


