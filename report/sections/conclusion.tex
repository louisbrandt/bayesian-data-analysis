\section{Conclusion}

\subsection{Limitations and Future Work}
The application of Bayesian Data Analysis in this study is an initial
exploration into the potential of Bayesian models in this case study. While the
models are informative, there were limitations to the study that should be
addressed in future work, the models to be more useful in industry.

Section \ref{subsec:missing-values} describes interpolation techniques to fill
in missing values in the data. Information is lost here; while interpolation
means the surrounding trend is preserved, reasoning explaining the
uncharacteristically low revenue is not offered. OLIOLI has had days of low
and no revenue and will again in the future, so being able to capture this
would give the models a more complete picture of the data. Flags for days of
interest, such as public holidays, would be an excellent way to capture this
information.

The Bayesian linear regression models struggled to predict accurately without
modelling the data as a time series, especially when given more historical
data. A possible explanation is a change in the underlying generative
process; the non-stationarity of the data. This aspect may explain some of
their underperformance. 

Whatever the angle, the task of modelling daily revenue in the real world is
difficult, due to the sheer number of factors that influence it. Weather and
calendar events are just two of these factors, and access to data for more
variables may result in a more accurate model of the underlying
revenue-generating process. Reducing the resolution of the data from daily to
hourly temperature and revenue updates may also be a way to improve the model.
However, this approach was not possible with the data available.

Being inherently uncertain, modelling uncertainty in the weather data itself
could be helpful, especially for multiday forecasting, allowing predictions to
reflect another source of uncertainty. This could be done by defining the
weather data as a random variable in the model.

Future studies may explore the optimal balance between data volume and
computational efficiency in more detail, especially as models become more
complicated and more features are added.

It would be valid to argue that there is a fundamental imprecision in the
modelling process. For example, the total revenue of a given day for the
company is determined by the total revenue for each store on that day. It is
possible that not modelling this relationship set a ceiling on the highest
possible accuracy of the model and was the primary reason behind the data's
non-stationarity. While modelling this relationship as a hierarchical model is
possible and encouraged by the Bayesian framework, \cite{statrethinking}
chapter 13, the required data pre-processing and model specification was beyond
the scope of this paper.

The hierarchical extension of the Bayesian linear regression and time-series
models to model the total revenue per store per day would give a more nuanced,
usable and interpretable insight into revenue for financial forecasting and
operational planning. 
In addition, the price of the product set by the company is also thought to
influence the revenue heavily. Incorporating this in a hierarchical
structure of models would open the door to other interesting modelling
approaches.
\subsection{Summary}
The present study explored various Bayesian models, including linear regression
and time-series models, for predicting daily revenue at OLIOLI, a
Copenhagen-based poke bowl chain. The investigation was motivated by industry
hypotheses suggesting a potential causal relationship between weather
conditions and revenue. This led to incorporating weather predictors in the
models, specifically temperature and precipitation. The results showed that
while larger data volumes and more autoregressive lags generally enhance model
performance, the addition of weather data only resulted in marginal
improvements in the predictive capacity of the models. Despite observing a
preliminary correlation between weather predictors and revenue, in correlation
analysis and more simple linear models, when added as predictors in
autoregressive models, the weather predictors did not improve the models'
predictive abilities. The best-performing model was still a hybrid model, which
combined weather data with 90 autoregressive lags, trained on the maximum
amount of historical data. The modest impact of weather data challenges the
initial industry assumptions and points to the potential for Bayesian
time-series models to be more suitable for daily revenue prediction. This study
sets the groundwork for additional Bayesian data analysis on this case study
and comparable scenarios. Future research should expand on the practical use of
Bayesian forecasting demonstrated here, emphasising its value in real-world
business decisions and resource allocation.
