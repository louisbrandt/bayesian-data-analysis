
\section{Data}
Revenue data from 2020-01-01 to 2023-03-21 with a consistent test set of
2023-03-22 to 2023-04-05.
Total in-store daily revenue for the company
Weather Data 

\subsection{Stationarity}
There are many stores that contribute to the generation of revenue, and over
the time period, the number of stores has increased from 7 to 15. By this
nature data != stationary. 
Why problem? 
not taking a variable with impact on the trend of the underlying data account
means spurious correlations are possible, which are relationships between
variables that appear to be significant but are, in fact, not. 
Some models assume stationarity
Options:
Include number of stores as a variable
Modelling each store as a separate model
Differencing data
Detrending data
Transformation

\subsection{Processing}
Normalise Target Variable
Normalise Continuous Variables

Categorical / Index variables
Index variables in bayesian models index a unique prior distribution for each
instantce of the variable

The Index variables
- Day of week
- Day in month
- Month of year
- Year since since 2020
- Precipitation since even distribution of 0 \& <1 \& >1

Choosing relevant weather variables

-- Missing values / outliers
On some days, store is closed. 
Problem because we need revenue data for each day CUZ.......
All days with revenue less than 60000 are replaced with the average of the surrounding days


\subsection{Exploratory Data Analysis}
Plots
Correlations
Time Series

\subsection{Data Split}
Depends on the question we want to answer
By default, we want to model the underlying trend of the data in order to approximate the data generation process to allow accurate prediction
Priority here is to forecast the future best, not explain the previous data best - similar but differnt questions.
The priority here is to forecast 3 weeks of revenue
Train: 2020-01-01 to 2023-03-21
Test: 2023-03-22 to 2023-04-05

