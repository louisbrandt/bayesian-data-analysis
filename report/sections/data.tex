\section{Data}

% notes
% Choosing relevant weather variables
% -- Missing values / outliers
% On some days, store is closed. 
% Problem because we need revenue data for each day CUZ.......
% All days with revenue less than 60000 are replaced with the average of the surrounding days
% Categorical / Index variables
% Index variables in bayesian models index a unique prior distribution for each
% instantce of the variable

The data for this case study is two-fold. The random variable of interest is
the total daily revenue of a restaurant chain - \texttt{OLIOLI} - and is a sum
of the revenue generated for each transaction retrieved and formatted via the
Point of Sale systems in each store. For each day, collected and forecasted
weather data is sourced from Visual Crossing - a leading provider of weather
data. The data set used for this project is populated with revenue figures and
weather metrics for each day from the 1st of January 2020 to the 23rd of April
2023.
% Data Description: Provide an overview of the dataset, including the number of observations, the time period covered, and the variables available. Explain the meaning of each variable and its relevance to your research question.

The data set contains 1209 observations, each representing a day and the following variables were collected

\begin{figure}[h!]
    \begin{subfigure}{0.5\textwidth}
      \centering
      \caption{Variable Source}
      \label{fig:vars1}
      \begin{tabular}{l}
          \toprule 
          \textbf{Weather variables} \\
          \quad\textit{Temperature} \\
          \quad\textit{Precipitation} \\
          \quad\textit{Wind speed} \\
          \quad\textit{Cloud cover} \\
          \quad\textit{Humidity} \\[0.3em]
          \textbf{Calendar variables} \\
          \quad\textit{Day of week} \\
          \quad\textit{Day of month} \\
          \quad\textit{Month} \\
          \quad\textit{Year} \\[0.3em]
          \textbf{Revenue variables} \\
          \quad\textit{Revenue} \\
          \quad\textit{Number of stores open} \\[0.3em]
          \bottomrule
      \end{tabular}
    \end{subfigure}%
    \begin{subfigure}{0.5\textwidth}
      \centering
      \caption{Variables in Model}
      \label{fig:vars2}
      \begin{tabular}{l}
          \toprule 
          \textbf{Continuous variables} \\
          \quad\textit{Temperature} \\
          \quad\textit{Wind speed} \\
          \quad\textit{Cloud cover} \\
          \quad\textit{Humidity} \\
          \quad\textit{Revenue} \\[0.3em]
          \textbf{Index variables} \\
          \quad\textit{Precipitation} \\
          \quad\textit{Day of week} \\
          \quad\textit{Day of month} \\
          \quad\textit{Month} \\
          \quad\textit{Year} \\
          \quad\textit{Number of stores open} \\[0.3em]
          \bottomrule
      \end{tabular}
    \end{subfigure}%
    \caption{Variables}
    \label{fig:variables}
\end{figure}

\subsection{Missing Values}

All values in the data that were below a threshold were deemed invalid; due to
unique circumstances like national holidays, COVID, shop closures, etc. In
order to fix this, interpolation was used to replace these numbers with data
from a timeframe that included 7 days before and 7 days after the problematic
value.

This method ensured the data was holistic and reliable while maintaining the
general structure and relationships between the variables. By employing this
window, any potential trends or patterns are captured and the replacement
numbers are consistent with the surrounding data.

Using machine learning algorithms or more complex imputation methods like
K-nearest neighbours imputation are other options for handling missing
variables. However, the interpolation approach used was thought to be
appropriate for this dataset and the objective, hence these alternatives were
not pursued.


\subsection{Standardisation}

Standardisation can be an important preprocessing step in Bayesian modelling,
as it assures that variables are on a comparable scale, enabling more efficient
sampling resulting in a more stable posterior. \cite{gelman2013philosophy}
In his paper \cite{gelman2004parameterization}, Gelman demonstrates how
reparameterization, including standardisation, can enhance the efficiency of
the sampling process in Bayesian models \cite{gelman2004parameterization}.
Continuous variables are normalised by scaling them to have a mean of 0 and a
standard deviation of 1, like so.

\begin{equation}
  x_{s} = \frac{x - \mu_x}{\sigma_x}
\end{equation}

The impact of extreme values in the target variable can also be reduced by
standardisation. To normalise the target variable, we apply the same formula as
for continuous variables.

Another important point is the consitency of this standardisation. Any new data
collected and procecessed must of course be standardised with the same
statistical metrics as the rest of the data to ensure accurate and coherent
inference and prediction. This was achieved by storing the mean and standard 
deviation of each variable in the data set, and applying the standardisation 
formula to new data using these values. Finally, to unstandardise the data, the 
reverse of the standardisation formula is applied, using the stored mean and
standard deviation. 

\begin{equation}
  x = x_{s} \cdot \sigma_x + \mu_x
\end{equation}


\subsection{Stationarity}
The nature of the problem introduces some caveats which need to be addressed in
the data. We know there are many stores that contribute to the generation of
revenue, and over the time period, the number of stores has increased from
seven to fifteen. As a company that grows and expands over time, the natural
and, for the company, ideal trend for the revenue to follow is upward. This
means that that the underlying data generating process changes - impacted by
the number of stores open at a given time, which season we are currently in and
percieved recognition and popularity of the company, to name a few factors. In
some cases, it is exactly this trend that we want to model and predict, but
doing so demands certainty of the isolation of the exact variables that impact
the trend. In practise, this was not feasable and some of the models developed
this paper, make some assumptions about the underlying data generating process.
One of these is the assumption of stationarity, which states that the
statistical properties of the data remain constant over time. We want to model
the data as a function of time, rather than a function of time and other
variables, avoiding the modelling of spurious correlations between variables
and permitting the modelling of the data as a time series, which is a common
approach to forecasting. 
%\cite{stationarity} \cite{time_series_stationarity} 

\subsubsection{Methods for resolving non-stationarity}

For resolving the issue of stationarity, we have a few options. Firstly, by
adding the number of stores open as an available variable, we can hope the
model will capture the cause of the increase in mean of the underlying data
generating process. This is a valid option that was implemented, and we will
discuss the implecations of taking such an approach, but it is not clear that
the number of stores is the only factor that influences the mean of the data
over time. 

Resolving non-stationarity can also be achieved by detrending the data. By
identifying and removing underlying trends from the data, it is possible to
lessen the impact of seasonality and long-term trends, highlight cyclical
patterns, and make it simpler to spot patterns and linkages in the data,
resulting in more accurate forecasts. 

One of these options is to not model the total revenue itself, but rather the
revenue difference between each day and the previous day to remove the linear
trend. This is known as differencing and is useful as the $n^{th}$ order
difference of non-stationarity is not necessarily non-stationary.

Furthermore, we know that Bayesian models have a robust framework in place for
modelling data in a hierarchical manner, and it would be equally valid to model
the problem as such. The total revenue is a sum of the revenue generated by
each store, and one of the benefits of taking such an approach would be
removing this specific aspect of non-stationarity from the problem. This
approach necessitates a far more vigorous approach to processing and modelling
the data and was out of scope for this project.

\subsubsection{Testing data for stationarity}
The notebook \texttt{stationarity\_testing.ipynb} is a thorough evaluation of
the stationarity of the data, the main findings of which will be detaied here. 

The test for stationarity performed was the Augmented Dickey-Fuller test
%\cite{adf_test}
. The null hypothesis of the test is that the data is non-stationary, and we
then check for the presence of a unit root. A time series with a unit root
indicates that the process that produced the data followed a random walk rather
than being stationary, which is due to the series' heavy reliance on its
historical values influencing the trends to drifts over time. Implementing the
test with the Python library \texttt{statsmodels} is simple; the results of
which can be seen in Table \ref{tab:adf_og}. 
\begin{table}[h]
\centering
\caption{ADF Results - Original Data}
\label{tab:adf_og}
\begin{tabular}{@{} >{\arraybackslash}l r @{}}
\toprule
\textbf{Test Statistic} & -2.1179 \\ \addlinespace[0.1em]
\textbf{P-Value} & 0.2374 \\ \addlinespace[0.1em]
\textbf{Critical Values} & \\ 
\ \ \ \ 1\% & -3.4359 \\ 
\ \ \ \ 5\% & -2.8640 \\ 
\ \ \ \ 10\% & -2.5681 \\ \addlinespace[0.1em]
\textbf{Information Criterion} & 1609.6593 \\ 
\bottomrule
\end{tabular}
\end{table}

Since the test statistic is greater than the critical values at all
significance levels, you cannot reject the null hypothesis that the time series
has a unit root (i.e. it is non-stationary). This conclusion is further
supported by the p-value of 0.2374, which is greater than the
standard scientific threshold 0.05.

\begin{table}[h]
\centering
\caption{ADF Results - First Difference}
\label{tab:adf_dif}
\begin{tabular}{@{} >{\arraybackslash}l r @{}}
\toprule
\textbf{Test Statistic} & -10.5442 \\ \addlinespace[0.1em]
\textbf{P-Value} & 8.4946e-19 \\ \addlinespace[0.1em]
\textbf{Critical Values} & \\ 
\ \ \ \ 1\% & -3.4359 \\ 
\ \ \ \ 5\% & -2.8640 \\ 
\ \ \ \ 10\% & -2.5681 \\ \addlinespace[0.1em]
\textbf{Information Criterion} & 1611.8563 \\ 
\bottomrule
\end{tabular}
\end{table}

Table \ref{tab:adf_dif} depicts the results of the test after taking the first
difference of the daily revenue data. we can see that the test statistic is now
less than the critical values at all significance levels, and the p-value is
now less than 0.05. This indicates that we can reject the null hypothesis -
that the time series has a unit root - and say the first difference is
stationary.

This is further supported by the plots of the data, which can be seen in the 
notebook. The original data shows an upward trend, which is removed by taking 
the first difference. The first difference also shows a more consistent 
variance over time, which is another requirement for stationarity.

%DO PLOTS 

By phrasing the question differently, it is possible to fulfill the requirement
of stationarity when it is needed.

\subsection{Exploratory Data Analysis}

Plots
Correlations
Time Series

\begin{table}[h]
\centering
\caption{Correlation Coefficients - Continuous Predictors}
\label{tab:correlation-numeric}
\begin{tabular}{@{} >{\arraybackslash}l r @{}}
\textbf{Variable} & \textbf{PCC} \\ \addlinespace[0.1em]
\toprule
\texttt{temperature} & 0.3341 \\
\texttt{humidity} & -0.3154 \\
\texttt{wind\_speed} & -0.2710 \\
\texttt{cloud\_cover} & -0.1628 \\
\bottomrule
\end{tabular}
\end{table}

From an initial exploration of the data, we can get a sense for the isolated
correlations of our various predictors and the target variable. Figure
\ref{tab:correlation-numeric} plots the linear relationships between the
normalised revenue and each of the normalised continuous predictors:
temperature, humidity, wind speed, and cloud cover. 

\begin{table}[h]
\centering
\caption{Kendall's Tau - Categorical Predictors}
\label{tab:correlation-categorical}
\begin{tabular}{@{} >{\arraybackslash}l r @{}}
\textbf{Variable} & \textbf{Tau} \\ \addlinespace[0.1em]
\toprule
\texttt{precipitation} & -0.1748 \\
\texttt{n\_stores} & 0.4688 \\
\texttt{dow} & 0.0051 \\
\texttt{day} & -0.0333 \\
\texttt{month} & -0.0578 \\
\texttt{year} & 0.5059 \\
\bottomrule
\end{tabular}
\end{table}

\subsection{Data Split}
The data split strategy employed in this study hinges on the specific research
question we aim to address. The primary objective is to accurately forecast
future revenues, with a focus on predicting revenue for a three-week period. In
light of this, the data is partitioned into training, validation, and test
sets.

The training set, spanning from \texttt{2020-01-01} to \texttt{2023-03-12},
serves to develop the model. The validation set, covering \texttt{2023-03-13}
to \texttt{2023-04-02}, facilitates model selection and tuning. Lastly, the
test set, ranging from \texttt{2023-04-03} to \texttt{2023-04-24}, assesses the
final model's performance on unseen data. Employing a three-week window for
validation and testing is a pragmatic choice, as real-world financial and
operational decisions often involve planning two-to-four weeks into the future.
This data split strategy ensures a fair evaluation of the model's predictive
ability, optimising it for future revenue forecasting.

Although Bayesian modelling inherently incorporates data into the model through
the likelihood function and lacks a separate 'out-of-sample', adopting a data
split strategy remains valuable. It provides a transparent and robust
assessment of the models' ability to generalise to new data, essential for
practical applications. Furthermore, it enables a fair comparison between
models, as they are all evaluated on the same held-out test set, allowing for a
more accurate assessment of their performance. Finally, the data split strategy
helps identify potential overfitting in the Bayesian context by evaluating the
models' performance on validation and test sets.

Given these considerations, implementing a data split strategy in this Bayesian
study is a valuable approach. It ensures a comprehensive evaluation of the
models' out-of-sample predictive accuracy and performance on different subsets
of data, enabling a robust assessment of the models' ability to predict the
restaurant chain's revenue. Coupling this data split strategy with Bayesian
model comparison techniques offers a thorough and reliable framework for model
evaluation and selection.
